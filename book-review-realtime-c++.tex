% Released under CC-BY-SA 2012
% Author: Stefan Petersen <spe@ciellt.se> 
%
% This presentation is based on the conference-ornate-20min template
% by Till Tantau <tantau@users.sourceforge.net>.  You may redistribute
% and/or modify it under the terms of the GNU Public License, version
% 2.
%
% To produce a PDF from this document, do something like
%
%   pdflatex FILE.tex
%
% You might want to have packages like tetex, latex-beamer and
% ghostscript installed.

% Presentation (C) 2013, Stefan Petersen (spe(a)ciellt.se)
% Denna presentation är licenserad under Creative Commons BY-NC-SA 3.0
% Attribution-NonCommercial-ShareAlike
% http://creativecommons.org/licenses/by-nc-sa/3.0/

\documentclass{beamer}
\mode<presentation>
{
  \usetheme{Warsaw}
  \setbeamercovered{transparent}
}
\usepackage[english]{babel}
\usepackage[utf8]{inputenc}
\usepackage{times}
\usepackage[T1]{fontenc}
\usepackage{url}
\usepackage{cclicenses}
%%%%%%%%%%%%%%%%%%%%%%%%%%%%%%%%%%%%%%%%
\title{Book recommendations }
\subtitle{Real-Time C++, Second Edition \\ \emph{Christopher Kormanyos}}
\author{Stefan Petersen}
\date[2018-10-25]{C++ Meetup 0x10, Stockholm, 2018 \\
  Released under Creative Commons BY-NC-SA 3.0 \\
  \cc \byncsa}
%\pgfdeclareimage[height=0.5cm]{ndn-logo}{images/DFRI-logo}
%\logo{\pgfuseimage{ndn-logo}}
%
%\AtBeginSection[]
%{
%  \begin{frame}<beamer>{Innehåll}
%    \tableofcontents[currentsection,currentsubsection]
%  \end{frame}
%}

% Enable this to make items appear one at a time.
%\beamerdefaultoverlayspecification{<+->}
%%%%%%%%%%%%%%%%%%%%%%%%%%%%%%%%%%%%%%%%
\begin{document}
\begin{frame}
  \titlepage
\end{frame}
\begin{frame}{Table of Contents}
  \tableofcontents
\end{frame}
%%%%%%%%%%%%%%%%%%%%%%%%%%%%%%%%%%%%%%%%
\section{Who am I and why do I care}
\begin{frame}{Who am I }
  \begin{itemize}
  \item Electronics designer and embedded programmer
  \item My own business since 2008, AB since 2012 (Ciellt AB)
  \item Programming microcontrollers since 1995
  \item Started open source project gerbv (gEDA)
  \item The quest for a ``safer C'' led me to C++ 1-2 years ago.
  \end{itemize}
\end{frame}
%%%%%%%%%%%%%%%%%%%%%%%%%%%%%%%%%%%%%%%%

%%%%%%%%%%%%%%%%%%%%%%%%%%%%%%%%%%%%%%%%
\section{Embedded microcontroller}

\begin{frame}{Embedded microcontrollers}
  \begin{itemize}
  \item  Limits in CPU speed
    \pause ==> most common 8-32 MHz
    \pause \item Limits in RAM
    \pause ==> 4 kb-384 kb (usually 10-16 kb)
    \pause \item Limits in flash (program memory)
    \pause ==> 16 kb-512 kb (usually 16-64 kb)
    \pause \item Simple, direct connection to hardware (lights, buttons etc)
    \pause ==> programming ``close to the metal''
    \pause \item Limited IO
    \pause ==> serial port if you're lucky, displays is special
  \end{itemize}
\end{frame}

\begin{frame}{What does this give?}
  \begin{itemize}
  \item Optimize program use
    \pause \item Use interrupts and DMA's
    \pause \item Precalculate as much possible, maybe tables?
    \pause \item Do not use dynamic memory
    \pause \item Avoid dynamic bindings
  \end{itemize}
\end{frame}

%%%%%%%%%%%%%%%%%%%%%%%%%%%%%%%%%%%%%%%%
\section{The Book}

\begin{frame}{Split into three sections}
  \begin{enumerate}[I]
  \item Language Technologies for Real-Time C++
  \item Components for Real-Time C++
  \item Mathematics and Utilities for Real-Time C++
  \end{enumerate}
\end{frame}

\begin{frame}{Language Technologies for Real-Time C++}
  \begin{itemize}
  \item Getting Started with Real-Time C++
  \item Working with a Real-Time C++ Program on a Board
  \item An Easy Jump-Start in Real-Time C++
  \item Object-Oriented Techniques for Microcontrollers
  \item C++ Templates for Microcontrollers
  \item Optimized C++ Programming for Microcontrollers
  \end{itemize}
\end{frame}

\begin{frame}{Components for Real-Time C++}
  \begin{itemize}
  \item Accessing Microcontroller Registers
  \item The Right Start
  \item Low-Level Hardware Drivers in C++
  \item Custom Memory Management
  \item C++ Multitasking
  \end{itemize}
\end{frame}

\begin{frame}{Mathematics and Utilities for Real-Time C++}
  \begin{itemize}
  \item Floating-Point Mathemathics
  \item Fixed-Point Mathemathics
  \item High-Performance Digital Filters
  \item C++ Utilities
  \item Extending the C++ Standard Library and the STL
  \end{itemize}
\end{frame}

\begin{frame}{Appendices}
  \begin{enumerate}[A]
  \item A Tutorial for Real-Time C++
  \item A Robust Real-Time C++ Environment
  \item Building and Installing GNU GCC Cross Compilers
  \item Building a Microcontroller Circuit
  \end{enumerate}
\end{frame}


%%%%%%%%%%%%%%%%%%%%%%%%%%%%%%%%%%%%%%%%

\section{Finale}

\begin{frame}{Conclusions}
\begin{itemize}
\item Good book if you want to use C++ on a microcontroller.
  \pause\item You have to have experience in microcontrollers
  \pause \item You have to have experience in programming these things.
\end{itemize}
\end{frame}


\begin{frame}{Other Sources}
  \begin{itemize}
  \item ``Getting Started with C++ in Embedded Systems'',
    Micharl Barr \& Michael Wilk (Barr Group), https://vimeo.com/170004397
  \end{itemize}
\end{frame}

\begin{frame}{Questions}
\begin{center}
  \Large Thank you! \normalsize\\
  \vspace{5mm}
  Real-Time C++, Second edition, \emph{Christopher Kormanyos} \\
  \tiny Efficient Object-Oriented and Template Microcontroller Programming \\
  \vspace{5mm}
  \normalsize Questions? \\
  \vspace{3mm}
  Contact: spe (at) ciellt (dot )se

\end{center}
\end{frame}


%%%%
\end{document}
